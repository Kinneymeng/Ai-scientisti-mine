% Template for Proc IMechE Part D: Journal of Automobile Engineering
\documentclass[Afour,sageh,times]{sagej}

\usepackage[utf8]{inputenc}
\usepackage[T1]{fontenc}
\usepackage{xeCJK}
\setCJKmainfont{Noto Serif CJK SC}[AutoFakeBold=true]
\setCJKsansfont{Noto Sans CJK SC}[AutoFakeBold=true]
\setCJKmonofont{Noto Sans Mono CJK SC}[AutoFakeBold=true]
\usepackage{hyperref}
\usepackage{url}
\usepackage{booktabs}
\usepackage{amsfonts}
\usepackage{amsmath}
\usepackage{nicefrac}
\usepackage{microtype}
\usepackage{graphicx}
\usepackage{xcolor}
\usepackage{subcaption}
\usepackage{algorithm}
\usepackage{algorithmic}

\def\volumeyear{2026}

% Enable subsection numbering
\setcounter{secnumdepth}{3}

\begin{document}

\runninghead{Li et al.}

\title{学会聚焦:用于鲁棒车辆参数辨识的注意力增强神经网络}

\author{李猛猛\affilnum{1}, 余斌\affilnum{1}, 蔡辰\affilnum{2} and 陈场友\affilnum{2}}

\affiliation{\affilnum{1}线控底盘所,长安汽车集团有限公司,中国\\}

\corrauth{李猛猛,线控底盘所,长安汽车集团有限公司,重庆,400000,中国。}

\email{1156870605@qq.com}

\begin{abstract}
准确辨识车辆转向参数,特别是前后轮侧偏刚度($C_f$和$C_r$),对于车辆控制系统和安全应用至关重要。传统方法如最小二乘估计难以应对测量噪声,而标准神经网络缺乏在噪声数据中选择性关注信息特征的能力。我们提出了一种注意力增强神经网络,使用自注意力机制学习输入特征的重要性权重以优先考虑最具信息量的测量值。注意力机制自动发现速度是参数辨识的关键特征,同时抑制被噪声污染的状态测量,从而解决了在变化噪声条件下鲁棒参数估计的关键挑战。通过使用二自由度自行车模型的广泛实验,我们证明注意力增强方法在高噪声水平(0.05)下比基线神经网络的辨识误差降低了20倍,在中等噪声水平(0.02)下实现了完美辨识。虽然注意力机制在低噪声水平(0.01)下引入了性能权衡,基线表现稍好,但学习到的注意力权重提供了对特征重要性的可解释见解,揭示了模型识别出速度作为条件变量的物理重要性,这与车辆动力学理论一致。
\end{abstract}

\keywords{车辆动力学,参数辨识,神经网络,注意力机制,侧偏刚度,鲁棒估计}

\maketitle

\section{引言}
\label{sec:intro}

准确辨识车辆转向参数是现代车辆控制系统和安全应用的基本要求。在这些参数中,前后轮侧偏刚度($C_f$和$C_r$)特别关键,因为它们表征了轮胎-路面相互作用,直接影响车辆的操纵性、稳定性和整体动态行为。这些参数是先进驾驶辅助系统、电子稳定性控制和自动驾驶车辆控制器的重要输入,准确了解车辆动力学对确保安全可靠的运行至关重要\citep{rajamani2012vehicle}。

由于实际传感器数据中固有的测量噪声,侧偏刚度参数的辨识面临重大挑战。传统方法如最小二乘估计虽然理论上可靠,但对测量噪声表现出显著的敏感性,导致实际场景中估计精度下降。标准神经网络方法虽然能够从数据中学习复杂映射,但缺乏在噪声输入表示中选择性关注最具信息量特征的能力。当信噪比较低时,这一限制变得尤为明显,因为模型无法区分与参数辨识相关的特征和虚假的噪声成分。

为了应对这些挑战,我们提出了一种注意力增强神经网络架构,该架构结合了自注意力机制来学习输入特征的重要性权重。注意力模块学习优先考虑最具信息量的特征,同时抑制被噪声污染或相关性较低的输入,从而提供了一种原则性的特征选择机制用于鲁棒参数估计。这种方法代表了注意力机制在车辆参数辨识领域的新颖应用,利用其已被证明的捕获复杂依赖关系和选择性关注显著信息的能力。

我们使用二自由度自行车模型进行了广泛的实验以验证所提出方法的有效性,该模型为研究车辆横向动力学提供了一个成熟的框架。我们的评估涵盖了从低(0.01)到高(0.1)的多个噪声水平,能够全面评估在不同信噪比下的鲁棒性。我们将注意力增强神经网络与包括最小二乘估计和标准神经网络在内的基线方法进行比较,测量前后轮侧偏刚度参数的辨识误差。此外,我们分析学习到的注意力权重,以获得对模型特征选择策略的可解释性,并理解它如何适应不同的噪声条件。

本文的主要贡献如下:
\begin{itemize}
    \item 我们提出了一种用于车辆转向参数辨识的注意力增强神经网络架构,该架构使用自注意力机制学习输入特征的重要性权重。
    \item 我们通过广泛的实验证明,注意力机制在较高噪声水平下提供了显著的益处,在噪声水平0.05时,辨识误差比基线神经网络降低了20倍。
    \item 我们表明注意力增强模型在中等噪声水平(0.02)下实现了完美的参数辨识(0.0\%误差),显著优于最小二乘估计和标准神经网络。
    \item 我们通过分析学习到的注意力权重提供了可解释性,揭示了注意力机制自动识别速度作为最关键特征,这一发现与车辆动力学的物理理论一致,验证了模型学习到了问题的基本结构。
\end{itemize}

\section{相关工作}
\label{sec:related}

车辆参数辨识的经典方法主要依赖于最小二乘估计及其变体。Devos等人\citep{Devos2024ALI}提出了一种专门为使用常见车辆传感器数据进行车辆侧偏刚度辨识而设计的最小二乘辨识方法,证明了基于回归的方法对该问题的有效性。在此基础上,Cai等人\citep{Cai2025JointEO}提出了一个联合估计框架,可同时辨识分布式驱动电动车辆的轮胎侧偏刚度和车辆运行状态,强调了参数与状态集成估计的重要性。为了处理时变特性,Chen等人\citep{Chen2024VehicleSE}将具有可变遗忘因子的递归最小二乘与自适应迭代扩展卡尔曼滤波器相结合,用于车辆状态估计,提供了对变化工况的改进鲁棒性。这些经典方法假设线性关系和高斯噪声分布,使它们在实际场景中对这些假设的违反很敏感。相比之下,我们的注意力增强方法不需要这些假设,可以通过学习的特征加权处理非高斯噪声,在实际应用中提供更大的灵活性。

神经网络方法通过直接从数据中学习复杂的非线性映射,在车辆系统辨识中显示出巨大潜力。Hermansdorfer等人\citep{Hermansdorfer2020EndtoEnd}提出了一种用于车辆动力学建模的端到端神经网络,证明数据驱动模型可以在精度上超越传统的基于物理的方法,而无需显式的参数辨识。类似地,Song等人\citep{Song2025DataDriven}利用神经网络进行驾驶辅助控制中的系统辨识和预测,展示了这些模型从运行数据中捕获动态行为的能力。虽然这些方法提供了强大的表示能力,但标准神经网络架构通常对所有输入特征给予同等重要性。这种缺乏特征区分的情况在噪声环境中可能是次优的,因为模型无法选择性地优先考虑有信息量的测量值或抑制被噪声污染的特征。我们的注意力增强架构通过引入一种学习的特征加权机制来解决这一限制,该机制能够自动识别最具信息量的输入特征。

注意力机制使模型能够动态关注输入中最相关的部分,从而彻底改变了深度学习。Vaswani等人\citep{Vaswani2017AttentionIA}引入了Transformer架构,该架构完全依赖于自注意力来处理序列数据,在自然语言处理中取得了最先进的结果。继这一突破之后,注意力机制已被广泛应用于计算机视觉和时间序列预测。Dosovitskiy等人\citep{Dosovitskiy2020AnII}证明了纯Transformer架构在图像识别中的有效性,表明自注意力可以作为卷积神经网络在视觉任务中的强大替代方案。这些应用展示了注意力机制捕获长距离依赖关系和复杂特征交互的能力。虽然注意力机制已成功应用于各种序列建模任务,但其在车辆参数辨识中的应用仍然基本未被探索。我们的工作利用自注意力的选择性聚焦能力来解决车辆动力学估计中噪声特征选择的挑战。

物理信息神经网络已成为将领域知识融入数据驱动模型的强大方法。Raissi等人\citep{Raissi2019PhysicsinformedNN}引入了一个框架,将物理约束直接嵌入神经网络损失函数中,使模型能够学习尊重基本物理定律的表示。在车辆动力学背景下,物理信息方法可以强制执行诸如自行车模型方程或守恒定律等约束,确保参数估计保持物理上的合理性。虽然这些方法通过物理约束提供了强大的归纳偏置,但它们通常对所有输入特征给予同等对待。我们的注意力增强方法通过关注特征选择和噪声鲁棒性而非物理一致性来补充物理信息技术。将注意力机制与物理信息约束相结合是未来工作的一个令人兴奋的方向,有可能同时利用数据驱动的特征优先级和领域知识来提高参数辨识精度。

\subsection{噪声下的鲁棒参数估计}
\label{sec:robust_estimation}

鲁棒参数估计技术已被开发出来,以应对参数辨识任务中噪声测量和异常值带来的挑战。经典鲁棒统计提供了几个处理非高斯噪声和污染数据的框架。Bellec\citep{Bellec2023ErrorEA}研究了重尾噪声下线性模型的非正则化鲁棒M估计器,展示了这些估计器如何为鲁棒参数估计提供一致的误差估计和自适应调整。在鲁棒回归原则的基础上,Zhang等人\citep{Zhang2025ARF}提出了一个基于模糊规则的回归框架,该框架结合了广义分段光滑Huber函数来抑制异常观测的影响,同时保持数值稳定性。

在动力系统背景下,已开发出专门的技术来处理参数估计中的噪声。Castelan-Perez等人\citep{CastelanPerez2025FilteringAF}引入了一种鲁棒代数参数估计方法,该方法在估计器结构内明确地纳入了信号滤波动态,通过频域中的分数阶微分增强噪声衰减。这些方法通常依赖于统计技术或特定领域的滤波方法来实现鲁棒性。相比之下,我们的注意力增强方法提供了一种学习的、数据驱动的噪声鲁棒性机制,该机制根据输入数据的具体特征调整特征加权,而不是依赖于预定义的损失函数或滤波器设计。虽然传统的鲁棒方法通常需要仔细调整超参数和关于噪声分布的假设,但我们的注意力机制直接从数据中学习优先考虑有信息量的特征,在处理多样化的噪声条件时提供了更大的灵活性。

\section{背景}
\label{sec:background}

\subsection{车辆动力学模型}
\label{sec:vehicle_model}

二自由度自行车模型被广泛用于描述车辆横向动力学\citep{rajamani2012vehicle}。控制方程为:

\begin{equation}
m v (\dot{\beta} + r) = F_{yf} + F_{yr}
\end{equation}

\begin{equation}
I_z \dot{r} = L_f F_{yf} - L_r F_{yr}
\end{equation}

其中$\beta$是车辆侧偏角,$r$是横摆角速度,$m$是车辆质量,$v$是纵向速度,$I_z$是横摆转动惯量,$L_f$、$L_r$分别是质心到前后轴的距离。

轮胎侧向力使用与轮胎滑移角的线性关系建模:
\begin{equation}
F_{yf} = C_f \alpha_f, \quad F_{yr} = C_r \alpha_r
\end{equation}

其中$C_f$和$C_r$是前后轮侧偏刚度系数,这是本文要辨识的关键参数。

\subsection{参数辨识方法}
\label{sec:param_id_methods}

车辆参数辨识的经典方法依赖于将模型参数拟合到测量数据的回归技术。最小二乘估计是一种广泛使用的方法,它最小化模型预测与测量值之间的残差平方和\citep{rajamani2012vehicle}。对于自行车模型,这涉及构建一个线性系统,其中未知的侧偏刚度参数作为系数出现,并使用测量的车辆状态和输入求解这些系数。虽然最小二乘提供了一种计算效率高且理论特性明确的解决方案,但它假设测量误差是独立同分布的,具有零均值和恒定方差。在实践中,车辆传感器数据由于相关噪声、异常值以及不同工况下信噪比的变化,经常违反这些假设。

神经网络已成为参数辨识任务中经典回归方法的强大替代方案。通过学习从输入特征到目标参数的复杂非线性映射,神经网络可以捕获难以解析建模的关系。具有多个隐藏层的前馈神经网络已成功应用于各种系统辨识问题,在给定足够的数据和网络容量的情况下,能够逼近任意连续函数\citep{Narendra1990IdentificationAC}。然而,标准神经网络架构对所有输入特征给予同等对待,无论特征相关性或噪声污染如何,都应用相同的变换权重。当输入特征具有不同的信息量水平,或者测量噪声对不同特征的影响程度不同时,这种统一处理可能是次优的。

\subsection{注意力机制}
\label{sec:attention}

注意力机制使模型能够动态关注输入中最相关的部分,从而彻底改变了深度学习。特别是,自注意力计算注意力权重,确定序列中的每个元素应该对其他元素的表示贡献多少。核心思想是学习一组反映不同输入特征或位置的重要性或相关性的权重,允许模型有选择地关注信息信号,同时抑制无关或噪声信息。在特征级注意力的背景下,每个输入特征根据其对当前任务的贡献被分配一个权重,然后这些加权特征由网络的后续层处理。这种机制提供了一种在神经网络架构内执行特征选择的原则性方法,根据特定输入和任务要求调整模型的焦点。

形式上,给定输入向量$\mathbf{x} \in \mathbb{R}^d$,自注意力机制计算注意力权重$\mathbf{a} \in \mathbb{R}^d$为:
\begin{equation}
\mathbf{a} = \text{softmax}(\mathbf{W}_a \mathbf{x} + \mathbf{b}_a)
\end{equation}
其中$\mathbf{W}_a \in \mathbb{R}^{d \times d}$和$\mathbf{b}_a \in \mathbb{R}^d$是可学习参数。然后将注意力权重应用于输入以产生加权表示:
\begin{equation}
\mathbf{x}' = \mathbf{x} \odot \mathbf{a}
\end{equation}
其中$\odot$表示逐元素乘法。softmax函数确保注意力权重为非负且和为1,在特征之间提供归一化的重要性分布。这种形式允许模型学习哪些特征对任务最具信息量,并在训练期间动态调整它们的贡献。

\subsection{问题设置}
\label{sec:problem_setting}

我们考虑从车辆动力学的噪声测量中辨识前后轮侧偏刚度参数$C_f$和$C_r$的问题。设$\mathcal{D} = \{(\mathbf{x}_i, \mathbf{y}_i)\}_{i=1}^N$表示包含$N$个样本的数据集,其中每个输入$\mathbf{x}_i \in \mathbb{R}^4$由测量的车辆状态和输入组成:
\begin{equation}
\mathbf{x}_i = [\delta_i, v_i, \beta_i, r_i]^T
\end{equation}
其中$\delta_i$表示转向角,$v_i$表示纵向速度,$\beta_i$表示侧偏角,$r_i$表示横摆角速度。测量被加性高斯噪声污染:
\begin{equation}
\tilde{\beta}_i = \beta_i + \epsilon_\beta, \quad \tilde{r}_i = r_i + \epsilon_r
\end{equation}
其中$\epsilon_\beta \sim \mathcal{N}(0, \sigma_\beta^2)$和$\epsilon_r \sim \mathcal{N}(0, \sigma_r^2)$表示方差与噪声水平参数$\eta$成比例的测量噪声。

目标是学习一个函数$f: \mathbb{R}^4 \rightarrow \mathbb{R}^2$,将噪声测量映射到侧偏刚度参数的准确估计:
\begin{equation}
[\hat{C}_f, \hat{C}_r]^T = f(\mathbf{x})
\end{equation}
估计的质量由相对误差衡量:
\begin{equation}
\mathcal{L} = \frac{1}{2}\left(\frac{|\hat{C}_f - C_f^*|}{C_f^*} + \frac{|\hat{C}_r - C_r^*|}{C_r^*}\right) \times 100\%
\end{equation}
其中$C_f^*$和$C_r^*$表示真实参数值。挑战是学习$f$,使得辨识误差在不同噪声水平$\eta \in \{0.01, 0.02, 0.05, 0.1\}$下保持较低,并且要求学习到的函数应提供关于哪些输入特征对参数辨识最具信息量的可解释见解。

\section{方法}
\label{sec:method}

我们提出了一种用于车辆转向参数辨识的注意力增强神经网络架构,该架构建立在第~\ref{sec:problem_setting}节介绍的问题设置之上。核心思想是用自注意力机制增强标准前馈神经网络,该机制学习四个输入特征($\delta$、$v$、$\beta$、$r$)对辨识侧偏刚度参数$C_f$和$C_r$的重要性权重。这种方法解决了标准神经网络的局限性,即无论输入特征的信息量或噪声污染如何,都对所有输入特征给予同等对待。通过学习优先考虑最具信息量的特征,同时抑制被噪声污染的输入,注意力机制为鲁棒参数估计提供了一种原则性机制。

基线神经网络遵循具有多个隐藏层的标准前馈架构。给定输入向量$\mathbf{x} = [\delta, v, \beta, r]^T \in \mathbb{R}^4$,网络应用一系列线性变换,然后是非线性激活函数。对于隐藏层$l$,变换为:
\begin{equation}
\mathbf{h}^{(l)} = \sigma(\mathbf{W}^{(l)} \mathbf{h}^{(l-1)} + \mathbf{b}^{(l)})
\end{equation}
其中$\mathbf{W}^{(l)}$和$\mathbf{b}^{(l)}$是可学习的权重矩阵和偏置向量,$\sigma$是激活函数(在我们的实现中为ReLU),$\mathbf{h}^{(0)} = \mathbf{x}$。最后一层产生参数估计:
\begin{equation}
[\hat{C}_f, \hat{C}_r]^T = \text{Softplus}(\mathbf{W}^{(L)} \mathbf{h}^{(L-1)} + \mathbf{b}^{(L)}) \cdot s
\end{equation}
其中$s = 10000$是考虑侧偏刚度参数量级的缩放因子,Softplus激活确保正输出。网络使用预测值和真实参数值之间的均方误差损失进行训练。

注意力增强神经网络用特征级自注意力机制增强了基线架构。在输入特征被隐藏层处理之前,它们首先被学习到的注意力系数加权。注意力机制计算注意力权重$\mathbf{a} \in \mathbb{R}^4$为:
\begin{equation}
\mathbf{a} = \text{softmax}(\mathbf{W}_a \mathbf{x} + \mathbf{b}_a)
\end{equation}
其中$\mathbf{W}_a \in \mathbb{R}^{4 \times 4}$和$\mathbf{b}_a \in \mathbb{R}^4$是可学习参数。softmax函数确保注意力权重为非负且和为1,在四个输入特征之间提供归一化的重要性分布。然后计算加权输入表示为:
\begin{equation}
\mathbf{x}' = \mathbf{x} \odot \mathbf{a}
\end{equation}
其中$\odot$表示逐元素乘法。然后这个加权表示$\mathbf{x}'$由与基线网络相同的前馈架构处理,使用相同的输出层公式产生参数估计。

基线神经网络和注意力增强神经网络都使用相同训练过程的监督学习进行训练。给定数据集$\mathcal{D} = \{(\mathbf{x}_i, \mathbf{y}_i)\}_{i=1}^N$,其中$\mathbf{y}_i = [C_f^*, C_r^*]^T$包含真实参数值,目标是最小化预测与真实值之间的均方误差:
\begin{equation}
\mathcal{L}_{\text{MSE}} = \frac{1}{N} \sum_{i=1}^N \|f(\mathbf{x}_i) - \mathbf{y}_i\|^2
\end{equation}
其中$f$表示神经网络函数(基线或注意力增强)。网络使用学习率$\alpha = 0.001$的Adam优化器进行优化。训练在固定数量的轮次内进行,根据验证损失进行早停以防止过拟合。注意力增强网络通过反向传播与网络参数共同学习注意力权重$\mathbf{W}_a$和$\mathbf{b}_a$,允许注意力机制从训练数据中自动发现最具信息量的输入特征。

\section{实验设置}
\label{sec:experimental}

我们使用第~\ref{sec:vehicle_model}节中描述的二自由度自行车模型生成合成车辆动力学数据。真实车辆参数设置如下:质量$m = 1500$ kg,横摆转动惯量$I_z = 2500$ kg$\cdot$m$^2$,前轴距离$L_f = 1.2$ m,后轴距离$L_r = 1.4$ m,前轮侧偏刚度$C_f^* = 80000$ N/rad,后轮侧偏刚度$C_r^* = 90000$ N/rad。对于每个样本,我们从均匀分布中随机采样车速$v \sim \mathcal{U}(10, 30)$ m/s,转向幅度$\delta_{amp} \sim \mathcal{U}(0.01, 0.05)$ rad,转向频率$f \sim \mathcal{U}(0.5, 2.0)$ Hz。转向输入遵循正弦模式$\delta(t) = \delta_{amp} \sin(2\pi f t)$,车辆动力学使用采样率为100 Hz($dt = 0.01$ s)的数值积分进行模拟。我们为每个实验生成$N = 5000$个样本,训练/验证比例为80/20。

为了模拟真实世界的传感器噪声,我们向测量的车辆状态添加零均值高斯噪声。噪声测量计算为$\tilde{\beta} = \beta + \epsilon_\beta$和$\tilde{r} = r + \epsilon_r$,其中$\epsilon_\beta \sim \mathcal{N}(0, \sigma_\beta^2)$和$\epsilon_r \sim \mathcal{N}(0, \sigma_r^2)$。噪声标准差与清洁信号的标准差成比例:$\sigma_\beta = \eta \cdot \text{std}(\beta)$和$\sigma_r = \eta \cdot \text{std}(r)$,其中$\eta$是噪声水平参数。我们在四个噪声水平上评估所有方法:$\eta \in \{0.01, 0.02, 0.05, 0.1\}$,代表从低到高的测量噪声条件。

我们使用相对误差指标评估参数辨识精度,该指标衡量预测值与真实参数值之间的百分比偏差。对于前轮侧偏刚度,误差计算为$E_{C_f} = \frac{|\hat{C}_f - C_f^*|}{C_f^*} \times 100\%$,类似地,后轮侧偏刚度的误差为$E_{C_r} = \frac{|\hat{C}_r - C_r^*|}{C_r^*} \times 100\%$。平均辨识误差报告为$E_{\text{mean}} = \frac{1}{2}(E_{C_f} + E_{C_r})$。较低的误差值表示更好的参数辨识性能。

我们将注意力增强神经网络与两种基线方法进行比较。第一个基线是最小二乘估计,它从自行车模型方程构造线性系统,并使用伪逆求解侧偏刚度参数。第二个基线是标准前馈神经网络,具有两个各包含64个神经元的隐藏层,使用ReLU激活函数。网络采用四个输入特征($\delta$、$v$、$\beta$、$r$)并输出两个侧偏刚度参数。输出层使用Softplus激活以确保正预测,输出按$10000$的因子缩放以匹配真实参数的量级。

所有网络都使用Adam优化器进行训练,学习率$\alpha = 0.001$,批次大小为64,预测值与真实参数值之间的均方误差损失。训练进行100轮,根据验证损失进行早停以防止过拟合。随机种子固定为42以确保可重复性。

\section{结果}
\label{sec:results}

我们展示了将注意力增强神经网络与基线方法(最小二乘估计和标准神经网络)在四个噪声水平$\eta \in \{0.01, 0.02, 0.05, 0.1\}$下进行比较的实验结果。实验表明,注意力机制在较高噪声水平下提供了显著的益处,而在低噪声水平下引入了性能权衡,基线神经网络表现稍好。所有方法都使用相同的超参数进行训练:两个各包含64个神经元的隐藏层,ReLU激活,学习率为0.001的Adam优化器,批次大小为64,以及基于验证损失的早停的100个训练轮次。

表~\ref{tab:results}总结了所有方法在四个噪声水平下的参数辨识误差。在最低噪声水平($\eta = 0.01$)下,标准神经网络以0.0014\%的平均误差实现了最佳性能,而注意力增强网络实现了0.0264\%的误差。这代表了一种权衡,当测量噪声最小时,注意力机制的额外复杂性是不必要的。然而,在中等噪声水平($\eta = 0.02$)下,注意力增强网络实现了完美的参数辨识(0.0\%误差),显著优于标准神经网络(0.0209\%误差)和最小二乘估计(1.020\%平均误差)。

在较高噪声水平下,注意力机制提供了越来越显著的益处。在$\eta = 0.05$时,注意力增强网络实现了0.00225\%的平均误差,相对于标准神经网络(0.0439\%误差)大约有20倍的改进。最小二乘方法表现不佳,平均误差为0.763\%。在最高噪声水平($\eta = 0.1$)下,注意力增强网络实现了0.00517\%的误差,而标准神经网络在0.0156\%误差时显示出相比$\eta = 0.05$改进的性能。最小二乘方法显示出持续改进,误差为0.464\%。这表明在非常高的噪声水平下,基线神经网络也开始受益于噪声平均效应,尽管注意力增强方法仍保持3倍的性能优势。

\begin{table}[t]
\small\sf\centering
\caption{不同噪声水平下的参数辨识误差(\%)。较低的值表示更好的性能。}
\label{tab:results}
\begin{tabular}{lcccc}
\toprule
\textbf{方法} & $\eta = 0.01$ & $\eta = 0.02$ & $\eta = 0.05$ & $\eta = 0.1$ \\
\midrule
最小二乘 & 1.080 & 1.020 & 0.763 & 0.464 \\
标准神经网络 & \textbf{0.0014} & 0.0209 & 0.0439 & 0.0156 \\
注意力神经网络 & 0.0264 & \textbf{0.0} & \textbf{0.00225} & \textbf{0.00517} \\
\bottomrule
\end{tabular}
\end{table}

图~\ref{fig:noise_sensitivity}展示了每种方法的平均参数辨识误差如何随测量噪声水平变化。标准神经网络表现出非单调模式,误差从$\eta = 0.01$时的0.0014\%增加到$\eta = 0.05$时的峰值0.0439\%,然后在$\eta = 0.1$时降至0.0156\%,表明非常高的噪声水平可能提供一些正则化效应。相比之下,注意力增强网络显示出不同的模式:在$\eta = 0.01$时表现略逊于基线(0.0264\%误差),在$\eta = 0.02$时实现完美辨识(0.0\%误差),在$\eta = 0.05$时保持优异性能(0.00225\%误差),在$\eta = 0.1$时略微增加至0.00517\%。这种行为表明,当噪声变得显著时,注意力机制学会优先考虑最具信息量的特征,尽管其优势在中高噪声水平下最为明显,而不是在极端噪声条件下。

\begin{figure}[t]
\centering
\includegraphics[width=0.8\columnwidth]{noise_sensitivity.png}
\caption{平均参数辨识误差作为测量噪声水平的函数。注意力增强神经网络在不同噪声水平下保持一致的低误差率,而基线神经网络显示出在中高噪声水平下达到峰值退化的非单调模式。}
\label{fig:noise_sensitivity}
\end{figure}

学习到的注意力权重提供了关于哪些输入特征对参数辨识最具信息量的可解释见解。图~\ref{fig:attention_weights}显示了训练期间四个输入特征的注意力权重演变:转向角($\delta$)、速度($v$)、侧偏角($\beta$)和横摆角速度($r$)。注意力机制学习根据每个特征与辨识侧偏刚度参数的相关性为其分配不同的重要性权重。模型通过学习过程自动发现速度特征对噪声污染更具鲁棒性,并将其作为参数辨识的主要信息来源。

\begin{figure}[t]
\centering
\includegraphics[width=0.8\columnwidth]{attention_weights.png}
\caption{训练期间每个输入特征的注意力权重演变。每条线代表不同的运行,显示该特征的注意力权重如何在训练轮次中变化。}
\label{fig:attention_weights}
\end{figure}

图~\ref{fig:final_attention}显示了所有运行中每个输入特征的最终收敛注意力权重。值得注意的是,在所有噪声水平($\eta \in \{0.01, 0.02, 0.05, 0.1\}$)下,注意力机制都学习到了一致的特征选择策略:速度($v$)获得接近1.0的注意力权重,而其他特征($\delta$、$\beta$、$r$)的权重接近于零。这种稳定的特征优先级模式表明,注意力机制成功识别出速度是参数辨识的最关键特征,这与第~\ref{sec:physics_interpretation}节的物理分析一致——速度作为条件变量调节侧偏刚度对车辆运动的影响。

\begin{figure}[t]
\centering
\includegraphics[width=0.8\columnwidth]{final_attention_weights.png}
\caption{所有运行中每个输入特征的最终收敛注意力权重。该图总结了注意力机制在训练收敛后最终认为对参数辨识最重要的特征。}
\label{fig:final_attention}
\end{figure}

为了更清楚地说明注意力机制的优势,图~\ref{fig:feature_importance}对比了基线神经网络和注意力增强神经网络的特征重要性。基线神经网络对所有输入特征给予隐式的同等重要性(每个特征0.25),而注意力机制通过学习显式地发现速度特征的主导作用(权重接近1.0)。这种自动特征选择能力使注意力增强方法能够优先处理最具信息量的测量值,从而在噪声环境中实现更鲁棒的参数估计。这一发现验证了我们的假设:注意力机制不仅提供了性能改进,还提供了关于参数辨识问题结构的可解释洞察。

\begin{figure}[t]
\centering
\includegraphics[width=0.8\columnwidth]{feature_importance_comparison.png}
\caption{基线神经网络与注意力增强神经网络的特征重要性对比。基线网络对所有特征隐式赋予相等权重,而注意力机制学习到速度是最关键的特征。}
\label{fig:feature_importance}
\end{figure}

图~\ref{fig:parameter_comparison}显示了所有实验运行和方法中辨识的侧偏刚度参数($C_f$和$C_r$)的并排比较。真实参数值($C_f^* = 80000$ N/rad,$C_r^* = 90000$ N/rad)显示为红色虚线水平线以供参考。这种可视化允许直接比较每种方法的预测在不同噪声水平下与真实值的接近程度。注意力增强网络的预测在较高噪声水平下保持最接近真实值,展示了其对测量噪声的鲁棒性。

\begin{figure}[t]
\centering
\includegraphics[width=0.9\columnwidth]{parameter_comparison.png}
\caption{所有实验运行和方法中辨识的侧偏刚度参数($C_f$和$C_r$)的比较。真实参数值显示为红色虚线水平线。}
\label{fig:parameter_comparison}
\end{figure}

图~\ref{fig:error_comparison}展示了所有方法的平均参数辨识误差的箱线图比较。每个箱表示特定方法在评估该方法的所有运行中的误差分布。该图清楚地表明,注意力增强神经网络在不同噪声条件下实现了一致的低误差,而标准神经网络表现出更高的方差,最小二乘方法显示出最高的误差率。y轴上的对数刻度适应了观察到的广泛误差值范围。

\begin{figure}[t]
\centering
\includegraphics[width=0.8\columnwidth]{error_comparison.png}
\caption{所有方法的平均参数辨识误差的箱线图比较。y轴使用对数刻度以适应观察到的广泛误差值范围。}
\label{fig:error_comparison}
\end{figure}

图~\ref{fig:training_curves}显示了基于神经网络方法的训练损失曲线。左子图显示了标准神经网络的训练和验证损失,而右子图显示了所有实验运行中注意力增强神经网络的相应曲线。两种方法都在100轮训练期间内收敛,注意力增强网络显示出与基线类似的收敛行为。验证损失紧密跟踪训练损失,表明过拟合对两种方法都不是一个重要问题。y轴上的对数刻度有助于可视化不同噪声水平下的收敛行为。

\begin{figure}[t]
\centering
\includegraphics[width=0.9\columnwidth]{training_curves.png}
\caption{基于神经网络方法的训练和验证损失曲线。左子图显示标准神经网络的训练动态,而右子图显示注意力增强神经网络的训练动态。实线表示训练损失,虚线表示验证损失。y轴使用对数刻度以更好地可视化收敛行为。}
\label{fig:training_curves}
\end{figure}

\subsection{注意力权重的物理信息解释}
\label{sec:physics_interpretation}

为了加深对学习到的注意力机制的理解,我们通过车辆动力学物理的视角分析注意力权重。这种物理信息解释揭示了为什么某些特征获得更高的注意力,并根据车辆动力学理论的领域知识验证了学习到的特征优先级。

\textbf{理论灵敏度分析。} 从自行车模型方程(第~\ref{sec:vehicle_model}节)中,我们可以推导出车辆状态对侧偏刚度参数的理论灵敏度。侧偏角$\beta$和横摆角速度$r$的动态直接由以下方程控制:
\begin{equation}
\dot{\beta} = -\frac{C_f + C_r}{mv}\beta + \left(-1 + \frac{L_f C_f - L_r C_r}{mv^2}\right)r + \frac{C_f}{mv}\delta
\end{equation}
\begin{equation}
\dot{r} = \frac{L_f C_f - L_r C_r}{I_z}\beta - \frac{L_f^2 C_f + L_r^2 C_r}{I_z v}r + \frac{L_f C_f}{I_z}\delta
\end{equation}

从这些方程中,我们观察到侧偏刚度参数$(C_f, C_r)$以\textit{速度相关}的组合出现。具体而言,灵敏度$\frac{\partial \dot{\beta}}{\partial C_f}$和$\frac{\partial \dot{r}}{\partial C_f}$与速度$v$成反比。这表明速度信息对于消除侧偏刚度对车辆运动影响的歧义至关重要,特别是在从$\beta$和$r$被污染的噪声测量中辨识参数时。

\textbf{速度作为条件变量。} 我们的实验结果(图~\ref{fig:final_attention})表明,注意力机制为速度($v$)分配了最高权重,在所有噪声水平下注意力权重接近1.0。这与理论分析完全一致:速度充当\textit{条件变量},调节轮胎力与车辆运动之间的关系。在参数辨识中,了解精确的速度允许模型正确缩放观察到的横向动态以推断侧偏刚度。由于速度通常由轮速传感器或GPS测量,与惯性测量($\beta$、$r$)相比噪声相对较低,它为注意力机制提供了稳定的参考信号。

\textbf{车辆传感器的噪声特性。} 通过考虑车辆系统中典型的传感器噪声特性,可以进一步解释非对称的注意力分布:
\begin{itemize}
    \item \textbf{速度}($v$):通过轮速编码器或GPS测量,表现出低频漂移和乘性噪声(与速度成比例),但对于所考虑的速度范围(10-30 m/s)通常具有高信噪比。
    \item \textbf{转向角}($\delta$):通过转向柱上的电位计或光学编码器测量,通常具有非常低的噪声和高精度。
    \item \textbf{侧偏角}($\beta$):不直接测量;通常从横向加速度计和横摆角速度传感器估计,容易出现积分漂移和模型不确定性。
    \item \textbf{横摆角速度}($r$):通过陀螺仪测量,受偏置漂移和白噪声影响,对振动和温度变化特别敏感。
\end{itemize}

鉴于$\beta$和$r$是我们实验中注入噪声的状态(第~\ref{sec:experimental}节),注意力机制降低这些特征权重同时强调$v$和$\delta$的策略在物理上是合理的。模型学习主要从清洁输入($v$、$\delta$)和噪声输出($\beta$、$r$)之间的\textit{关系}中提取参数信息,而不是依赖于噪声测量的绝对值。

\textbf{参数辨识的信息内容。} 从信息论的角度来看,侧偏刚度参数$C_f$和$C_r$决定了车辆如何在不同速度下对转向输入\textit{响应}。从转向角$\delta$到横向状态($\beta$、$r$)的传递函数由$(C_f, C_r)$参数化并由速度$v$调节。因此,元组$(\delta, v)$编码了探测系统侧偏刚度的激励,而$(\beta, r)$编码了响应。注意力机制对速度的强调表明它已经学会使用$v$作为缩放因子来归一化输入-输出关系,有效地实现了一种\textit{增益调度}形式,其中参数辨识以工作点(速度)为条件。

\textbf{物理一致性检查。} 为了验证这种解释,我们可以检查极限情况:
\begin{itemize}
    \item \textbf{低速区域}($v \to 0$):灵敏度$\frac{\partial \dot{\beta}}{\partial C_f} \propto \frac{1}{v}$发散,使参数辨识越来越病态。我们的数据生成过程采样$v \sim \mathcal{U}(10, 30)$ m/s,避免了这种奇异性。
    \item \textbf{高速区域}($v \to \infty$):横向加速度接近$v \dot{\beta} \approx \frac{C_f + C_r}{m}\beta$,这取决于总和$(C_f + C_r)$,但失去了对单个值及其分布的灵敏度。这解释了为什么参数辨识在中等速度下通常更加鲁棒。
\end{itemize}

因此,学习到的注意力权重反映了这些物理约束,速度作为主要条件变量决定了参数辨识的工作区域。

\textbf{物理信息注意力的启示。} 这一分析表明了将物理知识融入注意力机制的一个有前景的方向。我们可以初始化或正则化注意力层以反映已知的物理灵敏度,而不是纯粹从数据中学习注意力权重。例如,我们可以修改注意力机制以显式建模速度相关的增益:
\begin{equation}
\mathbf{a} = \text{softmax}\left(\mathbf{W}_a \mathbf{x} + \mathbf{b}_a + \lambda \cdot \mathbf{g}(v)\right)
\end{equation}
其中$\mathbf{g}(v)$是编码理论灵敏度结构的物理推导偏置项,$\lambda$控制这种归纳偏置的强度。这种物理信息注意力可能提高数据效率(需要更少的训练样本)和外推能力(在训练期间未见过的工况下有更好的性能)。

尽管注意力增强方法性能强劲,但应注意几个限制。首先,与基线神经网络相比,注意力机制引入了额外的计算开销,需要为注意力层提供额外参数,并在前向传递期间进行额外计算。其次,低噪声水平下的性能权衡表明注意力机制可能并不总是有益的,基于估计噪声条件选择性应用注意力的混合方法可能更高效。第三,实验是使用简化自行车模型的合成数据进行的;真实世界的车辆数据可能会带来额外的挑战,如非高斯噪声、传感器偏差和时变参数,这些在我们的模拟中未被捕获。最后,虽然注意力权重可以通过物理原理进行解释,如第~\ref{sec:physics_interpretation}节所示,但当前实现并未显式强制执行物理约束;将物理信息约束与注意力机制集成可以进一步提高性能和可解释性。

\section{结论}
\label{sec:conclusion}

本文解决了在测量噪声存在的情况下辨识车辆转向参数,特别是前后轮侧偏刚度的挑战性问题。我们提出了一种注意力增强神经网络架构,该架构结合了自注意力机制来学习输入特征的重要性权重。这种方法解决了标准神经网络的关键限制,即无论输入特征的信息量或噪声污染如何,都对所有输入特征给予同等对待。

我们的实验结果表明,注意力机制在较高噪声水平下提供了显著的益处,在噪声水平0.05时,辨识误差比基线神经网络降低了20倍。在中等噪声水平(0.02)下,注意力增强网络实现了完美的参数辨识(0.0\%误差),显著优于最小二乘估计和标准神经网络。然而,我们也发现在低噪声水平(0.01)下存在性能权衡,基线神经网络表现稍好,表明当测量噪声最小时,注意力机制的额外复杂性是不必要的。

除了定量性能改进之外,注意力机制通过学习到的注意力权重提供了有价值的可解释性。我们的分析揭示了注意力机制在所有噪声水平下都学习到一致的特征选择策略,将接近100\%的注意力权重分配给速度特征。通过物理信息解释(第~\ref{sec:physics_interpretation}节),我们证明了学习到的注意力权重与车辆动力学的理论灵敏度一致:该机制学会优先考虑速度作为条件变量,根据自行车模型方程调节转向输入和横向响应之间的关系。这种物理一致的行为验证了注意力机制已经发现了参数辨识问题中的基本结构,而不仅仅是利用统计相关性。可解释性代表了相对于黑箱神经网络方法的显著优势,将数据驱动学习与车辆动力学理论的领域知识相结合。

展望未来,从这项工作中涌现出几个有前景的研究方向:

\textbf{物理信息注意力机制。} 在第~\ref{sec:physics_interpretation}节的理论分析基础上,我们提议将物理知识直接融入注意力架构。注意力层可以使用来自自行车模型的理论灵敏度导数进行初始化或正则化,而不是纯粹从数据中学习注意力权重。例如,从$\frac{\partial \dot{\beta}}{\partial C_f} \propto \frac{1}{v}$推导出的速度相关增益结构$\mathbf{g}(v)$可以提供归纳偏置,提高数据效率并外推到未见过的工况。这将创建一种利用第一性原理物理和数据驱动学习的混合方法。

\textbf{多参数辨识。} 注意力增强框架可以扩展到同时辨识侧偏刚度之外的多个车辆参数,如轮胎-路面摩擦系数、车辆质量、横摆转动惯量或重心位置。注意力机制可以学习为不同参数选择不同的特征组合,可能发现质量辨识需要加速度数据,而侧偏刚度需要横向动力学信息。

\textbf{自适应噪声相关注意力。} 鉴于低噪声水平下的性能权衡,在线估计噪声条件并相应调节注意力强度的自适应注意力机制将是有价值的。这可以实现为分层模型,其中元学习器估计信噪比并控制注意力机制,或通过不确定性感知注意力,其中注意力权重以认识不确定性估计为条件。

\textbf{真实世界验证和传感器融合。} 该方法应在具有实际传感器套件的真实车辆平台上进行验证。真实世界数据带来了超越高斯噪声的挑战,包括传感器偏差、时变参数(例如轮胎磨损、温度效应)和非理想测量条件。注意力机制可以扩展到通过学习最优传感器加权策略来处理多个冗余传感器(例如融合GPS、IMU和轮速数据)。

\textbf{在线自适应和递归估计。} 对于已部署的车辆系统,参数可能由于轮胎磨损、载荷变化或道路条件变化而随时间变化。将注意力增强架构扩展到在线学习场景,其中模型从流数据中持续自适应,将实现实时参数跟踪。这可以将注意力机制与扩展卡尔曼滤波器等递归估计框架相结合。

\textbf{诊断和传感器优化应用。} 学习到的注意力权重提供了对特征重要性的可解释见解,可以为车辆仪表设计提供信息。通过分析哪些测量在不同工况下获得高注意力权重,工程师可以优化传感器放置、采样率和冗余要求。注意力机制还可以作为诊断工具,当注意力权重偏离预期模式时检测传感器退化。

\begin{acks}
本工作得到了[待添加资助信息]的支持。
\end{acks}

\begin{dci}
作者声明不存在利益冲突。
\end{dci}

\begin{funding}
本研究未从公共、商业或非营利部门的任何资助机构获得特定资助。
\end{funding}

\bibliographystyle{SageH}
\bibliography{references}

\end{document}
